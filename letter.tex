
\documentclass[12pt]{article}

\usepackage[margin=1in]{geometry} %Set margins
\usepackage{setspace} %Spacings

\pagestyle{empty} %No header/footer or page numbers

\newcommand{\signers}{
	\begin{table}[h]
	\raggedleft
	\begin{tabular}{rr}
		Tom Eichlersmith & Physics and Mathematics Major \\
		Ariya Ishida & Chemistry and Physics Major \\
		Opeyemi Arogundade & Physics and Chemistry Major \\
		Emily Anderson & Studio Art and Social Justice Flex Major
	\end{tabular}
	\end{table}
}

\newcommand{\opening}{
	\vskip 5mm
	\noindent
	An open letter to the staff, faculty, \\and members of the Board of Directors of Hamline:
	\vskip 5mm
}

\newcommand{\signatures}{
	\vskip 5mm
	\begin{table}
	\begin{tabular}{@{}p{0.5\textwidth}p{0.5\textwidth}@{}}
		\hrulefill & \hrulefill \\
		Tom Eichlersmith & Ariya Ishida \\
		& \\
		\hrulefill & \hrulefill \\
		Emily Anderson & Opeyemi Arogundade  \\
	\end{tabular}
	\end{table}
}

\begin{document}

%\onehalfspacing

\signers

\opening

We are a small but diverse group of fourth year students who wish to convey an important message to the people making long-term decisions for Hamline University.
This message is not meant to be an attack on your previous guiding decisions.
In fact, we wish to congratulate your group effort in constructing an environment here at Hamline that has allowed us to succeed in our respective fields.
We all hold several identities, but for the purposes of this message, we wish to point out that we are not only scientists, but musicians, artists, and writers.
We, no matter our connection with the natural sciences, have all been impacted by the collaborative undergraduate research (CUR) program here at Hamline, and wish to see it encouraged to grow.

The CUR program is unlike any other research programs available to students of our level throughout the nation.
Research programs at other universities do exist; nevertheless, they are confined to the natural sciences and they do not offer the independence that truly allows the student to learn.
The CUR program supports both of these rare advancements in undergraduate education, which makes Hamline a premier institution for students to \textbf{not only \emph{do} research but to \emph{learn how} to research}.
We fear that the uniqueness of the CUR program here at Hamline is under-appreciated, and we wish to make a public statement in support of it.

Hamline has allowed students to follow several different avenues towards participating in research, and this flexibility encourages students to pursue interesting research about which they are passionate.
We have gained funding from outside sources (such as the Howard-Hughes Medical Institute), the summer collaborative undergraduate research, and our departments directly.
Recognizing that Hamline is not financially capable of supporting every student in research, allowing several paths to funding gives students who are passionate about their work several opportunities to fund it.
Students who then are able to do their research are able to not only learn from the research itself but also learn from the process of searching for funding which is sporadic and daunting.
This combined process of searching for funding and then researching with the obtained funding gives students several skills that help them not only in academia but in any pursuit where an appeal to authority --- the people with the money --- is required.
It is in our estimation then that encouragement of research at Hamline includes support for searching for external funding as well as more financial support internally.


\signatures

\end{document}